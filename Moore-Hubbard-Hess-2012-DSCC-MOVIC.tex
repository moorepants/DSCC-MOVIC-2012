%%%%%%%%%%%%%%%%%%%%%%%%%%% asme2e.tex %%%%%%%%%%%%%%%%%%%%%%%%%%%%%%%
% Template for producing ASME-format articles using LaTeX            %
% Written by   Harry H. Cheng                                        %
%              Integration Engineering Laboratory                    %
%              Department of Mechanical and Aeronautical Engineering %
%              University of California                              %
%              Davis, CA 95616                                       %
%              Tel: (530) 752-5020 (office)                          %
%                   (530) 752-1028 (lab)                             %
%              Fax: (530) 752-4158                                   %
%              Email: hhcheng@ucdavis.edu                            %
%              WWW:   http://iel.ucdavis.edu/people/cheng.html       %
%              May 7, 1994                                           %
% Modified: February 16, 2001 by Harry H. Cheng                      %
% Modified: January  01, 2003 by Geoffrey R. Shiflett                %
% Use at your own risk, send complaints to /dev/null                 %
%%%%%%%%%%%%%%%%%%%%%%%%%%%%%%%%%%%%%%%%%%%%%%%%%%%%%%%%%%%%%%%%%%%%%%

%%% use twocolumn and 10pt options with the asme2e format
\documentclass[twocolumn,10pt]{asme2e}

%% The class has several options
%  onecolumn/twocolumn - format for one or two columns per page
%  10pt/11pt/12pt - use 10, 11, or 12 point font
%  oneside/twoside - format for oneside/twosided printing
%  final/draft - format for final/draft copy
%  cleanfoot - take out copyright info in footer leave page number
%  cleanhead - take out the conference banner on the title page
%  titlepage/notitlepage - put in titlepage or leave out titlepage
%
%% The default is oneside, onecolumn, 10pt, final

%% Replace here with information related to your conference
\confshortname{DSCC 2012} \conffullname{2012 ASME Dynamic Systems and Control
Conference} \confdate{17-19} \confmonth{October}
\confyear{2012} \confcity{Fort Lauderdale, Florida} \confcountry{USA}

%% Replace DETC2005-12345 with the number supplied to you
%% by ASME for your paper.
\papernum{DSCC2008-12345}

\title{DRAFT: System identification of a bicycle under manual control and
lateral perturbations}

\author{Jason K. Moore\thanks{Address all correspondence to this author.}\\
        {\tensfb Mont Hubbard}\\
				{\tensfb Ronald Hess}
    \affiliation{
    Department of Mechanical and Aerospace Engineering\\
    University of California\\
    Davis, California 95616\\
    Email: jkmoor@ucdavis.edu
    }
}

\usepackage{amsmath}
\usepackage{graphicx}

\begin{document}
%\pagestyle{empty}

\maketitle

\begin{abstract}
	\textit{We compare two bicycle models derived from first principles to a set
	of empirically derived models. The empirical models were developed with
	structured state space system identification techniques. The data was
	collected during a series of experiments on a gymnasium floor and a large
	treadmill. During the experiments, three rigidified bicycle riders manually
	controlled the bicycle through steering control for two tasks: heading
	tracking and lateral deviation tracking. Both tasks where performed with and
	without measured lateral pulsive perturbations. The bicycle was instrumented
	to accurately measure the all of the essential kinematics and kinetics
	involved. The resulting the rider control actions excited the system in a
	bandwidth adequate for identifying the linear dynamics of the system. The
	resulting models are compared with several archetypal first principal models
	showing that the simplest bicycle models are not necessarily sufficient for
	capturing the fundamental dynamics of the vehicle. The roll acceleration
	equation was found to be reasonably predicted by the simpler first principle
	bicycle models, but the steer acceleration equation was not. A model which
	includes the inertial effects of the rider’s arms, which more closely modeled
	reality, improved the model predictions, but still lacked full fidelity. We
	conclude the paper by exploring the validity of several of the modeling
	assumptions and potential sources for the prediction error.}
\end{abstract}

\section*{INTRODUCTION}

There is a rich history of bicycle and motorcycle mathematical model
development which has been able to explain many of the more dynamically
fascinating phenomena from countersteering and stability to speedman's wobble
and gyroscopic effects, see \cite{Limebeer2006, Meijaard2007} for historical
overviews, but the amount of experimental validation of these idealized models
pales in comparison, with the motorcycle experimentation just outdistancing
that done with bicycles.

The earliest comprehensive bicycle model validation began at CALSPAN in the
late 60's \cite{Roland1971}. They had several revolutionary studies, one in
which they made use of a rocket to apply know step torques to an uncontrolled
riderless bicycle and simulations of slalom maneuvers which was visually
compared with video footage the maneuvers. Little more was done done until more
recently. Kooijman \cite{Kooijman2006, Kooijman2009} made use of an
uncontrolled riderless bicycle's stability and excited the weave mode showing
good agreement with the eigenvalues of linearized Whipple model. Lastly, on the
theoretical side, Chen and Doa \cite{Chen2010} make use of a structured black
box identification of simulated noisy data and were able to basically reproduce
the eigenvalues of a Whipple model from simulated data.

In terms of motorcycle model validation, David Eaton's dissertation work in
particular stands out. He identified a motorcycle with both spectral techniques
and basic regression techniques in his 1973 dissertation \cite{Eaton1973}. He
did experiments with an uncontrolled motorcycle with rigidified rider and
excited the system by tapping the handlebars and releasing weights for a change
in roll torque. He showed good prediction of the weave and capsize modes.  He
was also able to estimate the motorcycle model from the spectral content of the
measured steer torque and roll angle while the motorcycle was manually
controlled. Eaton's methods in many ways laid the groundwork for many
subsequent studies by other researchers, including this one.

Stephen James \cite{James2002} identified an off-road motorcycle which was
manually excited by the rider by using black box ARX methods. He compared the
experimental results with a first principle motorcycle model he developed and
claimed good agreement based on visual interpretation of the root loci plots.
Biral et al. \cite{Biral2003} constructed empirical Bode plots of a motorcycle
under manual control from steer torque and kinematic data during a series of
slalom tests. They were able to get very good agreement with their motorcycle
model. Doria et al. \cite{Doria2012} excited the weave mode of a motorcycle at
high speed with a handlebar tap by the rider and was able to show agreement to
their predictions through a phase portrait view.

Basic bicycle and motorcycle identification is typically done by exciting the
vehicle through either perturbations in roll or steer. These experiments can be
done when the bicycle is under closed loop control or when the bicycle is
stable, the later being a requirement for speeds outside of the stable speed
range. But, the mode excitation methods are limited to the frequency band
around that mode of motion. Manual excitation under closed loop control gives
better excitation bandwidth and a pairing with modern system identification
techniques can provide richer models.

\section*{MEASUREMENTS}
We constructed an instrumented bicycle with a wide variety of kinematic and
kinetic sensors. The rider's torso was rigidified with respect to the bicycle
frame with the use of a torso cast. The rider's legs were fixed to the frame
with magnetic knee straps and an electric motor provided steady propulsion. The
rider's head and arms were able to move, the latter for steering control. Great
care was taken to measure steer torque as accurately as possible. We made use
of a 17 N-m single axis torque sensor mounted such that it was impossible to
apply any forces or moments besides the steer torque. The steer angle and roll
angle were measured with potentiometers, the later through the roll rotation of
a lightweight trailer. The angular rates of the bicycle frame and the
acceleration of a point on the frame were measured with an inertial measurement
unit. The body fixed rate of the front frame about the steer axis was measured
with a rate gyro. The lateral perturbations were delivered by a push/pull rod
with a 450 N load cell.

The raw voltage signals were collected on-board via a data acquisition unit and
a netbook laptop with a Matlab interface. A processing program, was developed
to create and manage an HDF5 database of all of the signal data (700+ runs) and
accompanying metadata for each run. The software allows for easy querying,
converts raw voltage signals to the properly scaled engineering units, computes
the desired quantities for model comparison, and filters and truncates the
signals depending on the task for selecting the best data for identification
purposes. The black line in Figure \ref{fig:exampleFit} shows an example of
the processed experimental data for one run.

\begin{figure}
	\includegraphics{figures/example-fit.pdf}
	\caption{The example results for the identification of a single run. The
	experimentally measured steer torque and lateral force are shown in the top
	two graphs. All of the signals were filtered with a 2nd order 15 hertz low
	pass Butterworth filter and the means were subtracted. The remaining four
	graphs show the simulation results for the Whipple model (W), Whipple model
	with the arm inertia (A), and the identified model for that run (I) plotted
	with the measured data (M). The percentages give the percent of variance
	explained by the model.}
	\label{fig:exampleFit}
\end{figure}

\section*{EXPERIMENTS}
We collected data with three riders of similar age, stature, and proficiency in
bicycling for a series of experiments. The data presented herein focuses on
two tasks:

\begin{description}
	\item[Heading Tracking]
		The rider was instructed to simply balance the bicycle and keep a
		relatively straight heading while focusing their vision at a point in the
		far distance.
	\item[Lateral Deviation Tracking]
		The rider was instructed to focus on a straight line that was marked on the
		ground and to attempt to keep the front wheel on the line.
\end{description}

Both tasks were performed with and without the application of a manually
applied lateral perturbation force just below the seat. The forces were applied
randomly in direction and time.

We performed the experiments on a treadmill and in a large gymnasium. The
treadmill is 1 meter wide by 5 meters long and has a speed range from 0.5 m/s
to 17 m/s. The treadmill was narrow but after some practice we were able to
successfully perform the lateral perturbations experiments. This provided a
very controlled environment, in particular with respect to fixed speeds and
allowed for very long runs durations within a broad speed range. The gymnasium
floor provided a rectangular wind free space of about 30 m by 55 m. We rode
around the perimeter to build up speed and did our maneuvers on a straight
section about 30 m long. We were not able to travel at speeds higher than about
7 m/s due to the tires slipping in the final turn into the test section.

\section*{FIRST PRINCIPLE MODELS}
The passive dynamics of the bicycle-rider system can be described with many
models. The benchmarked \cite{Meijaard2007,Basu-Mandal2007} Whipple model
\cite{Whipple1899} provides a somewhat minimalistic model capable of capturing
the essential dynamics of a bicycle-rider system. We use this model as the
standard base model to work from, as the fidelity of simpler models are
generally not adequate. The model is 4th order with roll angle, steer angle,
roll rate and steer rate as the states and typically a roll torque and steer
torque as the inputs. We neglect the roll torque input and in its place extend
the model such that a lateral force acting at a point on the frame provides a
new input, to accurately model the lateral perturbations. The second model that
we consider adds the inertial effects of the rider's arms because during the
experiments the riders were free to move their arms along with the front frame
of the bicycle. This extension is similar in fashion to the upright rider in
\cite{Schwab2012}, except with slightly different joint definitions.
Constraints are then chosen such that no additional degrees of freedom are
added keeping the system both tractable and comparable to the benchmarked
Whipple model.

We populate the linear model coefficients using measured parameters of the
bicycle and rider. The bicycle was measured as described in \cite{Moore2010} to
get accurate estimates of the parameters used in the benchmark bicycle.  Each
rider's inertial properties were estimated with Yeadon's method
\cite{Yeadon1990a} and allowed easy extraction of body segment parameters for
more complicated rider biomechanic models such as the inclusion of moving arms
as described above. The parameter computation is handled with a two custom open
source software packages \cite{Moore2011a,Dembia2011}.

\section*{MODEL IDENTIFICATION}
Each run provided a collection of time series that describe the various inputs
and outputs of the system. For this analysis, we limit our inputs to steer
torque and lateral force and our outputs to roll angle, steer angle, roll rate
and steer rate. This ideal fourth order system can be described by the
following continuous state space description
\begin{equation}
	\begin{split}
		\dot{x}(t) & =
		\mathbf{F}x(t) + \mathbf{G}u(t)\\
		\begin{bmatrix}
			\dot{\phi} \\
			\dot{\delta} \\
			\ddot{\phi} \\
			\ddot{\delta}
		\end{bmatrix}
		& =
		\begin{bmatrix}
			0 & 0 & 1 & 0\\
			0 & 0 & 0 & 1\\
			a_{\ddot{\phi}\phi} & a_{\ddot{\phi}\delta} &
			a_{\ddot{\phi}\dot{\phi}} & a_{\ddot{\phi}\dot{\delta}}\\
			a_{\ddot{\delta}\phi} & a_{\ddot{\delta}\delta} &
			a_{\ddot{\delta}\dot{\phi}} & a_{\ddot{\delta}\dot{\delta}}
		\end{bmatrix}
		\begin{bmatrix}
			\phi \\
			\delta \\
			\dot{\phi} \\
			\dot{\delta}
		\end{bmatrix}
		+
		\begin{bmatrix}
			0 & 0 \\
			0 & 0\\
			b_{\ddot{\phi}T_\delta} & b_{\ddot{\phi}F_{c_l}}\\
			b_{\ddot{\delta}T_\delta} & b_{\ddot{\delta}F_{c_l}}
		\end{bmatrix}
		\begin{bmatrix}
			T_\delta\\
			F_{c_l}
		\end{bmatrix}\\
		\eta(t) & = \mathbf{H}x(t)\\
	\end{split}
\end{equation}
where $\mathbf{H}$ is the identity matrix.

Assuming that this model structure can adequately capture the dynamics of
interest in the bicycle/rider system, our goal is to accurately identify the
unknown parameters $\theta$ which are made up of the unspecified entries in the
$\mathbf{F}$ and $\mathbf{G}$ matrices. This continuous formulation is not
compatible with noisy discrete data and the following difference equation can
be assumed if we sample at $t=kT$, $k=1,2,\dots$ and that the values are
constant over the sample period.

\begin{equation}
	\begin{split}
		x(kT + T) & = \mathbf{A}(\theta)x(kT) + \mathbf{B}(\theta)u(kT) + w(kT)\\
		y(kT) & = \mathbf{C}(\theta)x(kT) + v(kT)
	\end{split}
\end{equation}

The additional terms, $w$ and $v$ are the process noise and the measurement
noise, respectively, which are assumed to be white and Gaussian with zero mean
and known variance. In the analysis presented here, we assume the process
noise, $w$, is zero.

We made use of the Matlab System Identification Toolbox for the identification
of the parameters $\theta$ in each run of this output error model structure.

Figure \ref{fig:exampleFit} shows the example input and output data for a
single run with both steer torque and lateral force as inputs. Notice that the
identified model predicts the trajectory extremely well and similar results are
found for the majority of the runs. The Whipple model predicts the trajectory
directions but the magnitudes are too large, meaning that for a given
trajectory, the Whipple model requires less torque than what was measured. The
extended Whipple model with the arm inertial effects does a better job than the
Whipple model, but still has some magnitude differences.

\section*{MODEL COMPARISON}
We run the identification procedure described in the previous section for a
collection of runs ($n=368$) over a range of speeds between about 1 and 9 m/s.
Figure \ref{fig:coefficients} plots the identified coefficients of the
dynamical equations of motion (i.e. the bottom two rows of the \(\mathbf{F}\)
and \(\mathbf{G}\) matrices) as a function of speed for each run as compared to
both the Whipple model prediction and the arm model prediction. From visual
inspection the $\ddot{phi}$ equation seems to be better predicted by the two
models. The roll equation also seems to have less spread in the experimental data. For
example, the \(a_{\ddot{\phi}\delta}\) coefficient appears to be very tight and the
first principles models predict it very well. The constant, linear and
quadratic trends in the coefficients are not easily seen for most coefficients
due to the variability in the high number of repeated experiments at some speeds.

Figure \ref{fig:bode} gives another view of the resulting data. It is a
frequency response plot at the mean speed for a set of runs. The blue lines
give the mean and standard deviation of the magnitude and phase of the system
transfer function \(\frac{\phi}{T_\delta}(s)\). Even though the spread in
the identified parameters seems high in Figure \ref{fig:coefficients}, the Bode
plot shows that the system response is not as variable, especially in magnitude.
It is also apparent that the experimental magnitude has a -5 to -10 dB offset
across the frequency range shown with respect to the Whipple model. This
correlates with the amplitude differences in the trajectories shown in
\ref{fig:exampleFit}. Notice that the arm model has little to no offset between
2 and 10 rad/s, thus the better amplitude matching.

\begin{figure*}
	\includegraphics{figures/coefficients.pdf}
	\caption{The coefficients of the linear dynamical equations of motion plotted
	as a function of speed. Each blue dot is a single experiment. The green line
	is the Whipple model and the red line is the arm model. Only experiments
	with a mean fit percentage greater than zero are shown.}
	\label{fig:coefficients}
\end{figure*}

\begin{figure}
	\includegraphics{figures/bode.pdf}
	\caption{The frequency response of the steer torque to roll angle transfer
	function for a set of runs at $4.0 \pm 0.3$ m/s. The solid blue line is the
	mean from the identified runs and is bounded by the standard deviation, the
	dotted blue line. The green line is the Whipple model and the red line is the
	from the arm extension.}
	\label{fig:bode}
\end{figure}

\section*{CONCLUSION}
We have shown that a fourth order structured state space model is adequate for
describing the motion of the bicycle under manual control in a speed range from
approximately 1.5 m/s to 9 m/s. The fact that higher order models may not be
necessary for bicycle dynamic description is an important finding, but the
data subsequently reveals that archetypal first principles models are not
robust enough to fully describe the dynamics. The deficiencies are most
probably due to the knife edge, no side slip assumptions made in the two
models. The uncertainty in our first principle model physical parameter estimates
is not likely large enough to cause the them to better predict the data. It is
more likely that a tire scrub torques of some sort are needed to improve the
fidelity. Future work will focus on identifying the process noise for better
parameter estimation, fitting regression models to the coefficient data, and
adding simple tire scrub torque extensions to the first principle model.

\begin{acknowledgment}
	This material is based upon work supported by the National Science Foundation
	under Grant No 0928339.
\end{acknowledgment}

\bibliographystyle{asmems4}
\bibliography{bicycle}

\end{document}
