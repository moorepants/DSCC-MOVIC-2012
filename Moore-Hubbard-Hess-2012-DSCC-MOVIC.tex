%%%%%%%%%%%%%%%%%%%%%%%%%%% asme2e.tex %%%%%%%%%%%%%%%%%%%%%%%%%%%%%%%
% Template for producing ASME-format articles using LaTeX            %
% Written by   Harry H. Cheng                                        %
%              Integration Engineering Laboratory                    %
%              Department of Mechanical and Aeronautical Engineering %
%              University of California                              %
%              Davis, CA 95616                                       %
%              Tel: (530) 752-5020 (office)                          %
%                   (530) 752-1028 (lab)                             %
%              Fax: (530) 752-4158                                   %
%              Email: hhcheng@ucdavis.edu                            %
%              WWW:   http://iel.ucdavis.edu/people/cheng.html       %
%              May 7, 1994                                           %
% Modified: February 16, 2001 by Harry H. Cheng                      %
% Modified: January  01, 2003 by Geoffrey R. Shiflett                %
% Use at your own risk, send complaints to /dev/null                 %
%%%%%%%%%%%%%%%%%%%%%%%%%%%%%%%%%%%%%%%%%%%%%%%%%%%%%%%%%%%%%%%%%%%%%%

%%% use twocolumn and 10pt options with the asme2e format
\documentclass[twocolumn,10pt]{asme2e}

%% The class has several options
%  onecolumn/twocolumn - format for one or two columns per page
%  10pt/11pt/12pt - use 10, 11, or 12 point font
%  oneside/twoside - format for oneside/twosided printing
%  final/draft - format for final/draft copy
%  cleanfoot - take out copyright info in footer leave page number
%  cleanhead - take out the conference banner on the title page
%  titlepage/notitlepage - put in titlepage or leave out titlepage
%
%% The default is oneside, onecolumn, 10pt, final

%% Replace here with information related to your conference
\confshortname{DSCC 2012}
\conffullname{2012 ASME Dynamic Systems and Control Conference}
\confdate{17-19}
\confmonth{October}
\confyear{2012}
\confcity{Fort Lauderdale, Florida}
\confcountry{USA}

%% Replace DETC2005-12345 with the number supplied to you
%% by ASME for your paper.
\papernum{DSCC2008-12345}

\title{System identification of a bicycle under manual control and lateral
perturbations}

\author{Jason K. Moore\thanks{Address all correspondence to this author.}
    \affiliation{
    Sports Biomechanics Laboratory\\
    Department of Mechanical and Aerospace Engineering\\
    University of California\\
    Davis, California 95616\\
    Email: jkmoor@ucdavis.edu
    }
}

\author{Mont Hubbard
    \affiliation{
    Sports Biomechanics Laboratory\\
    Department of Mechanical and Aerospace Engineering\\
    University of California\\
    Davis, California 95616\\
    Email: mhubbard@ucdavis.edu
    }
}

\author{Ronald Hess
    \affiliation{
    Department of Mechanical and Aerospace Engineering\\
    University of California\\
    Davis, California 95616\\
    Email: rahess@ucdavis.edu
    }
}

\author{Ronald Hess
    \affiliation{
    Department of Mechanical and Aerospace Engineering\\
    University of California\\
    Davis, California 95616\\
    Email: rahess@ucdavis.edu
    }
}

\begin{document}
%\pagestyle{empty}

\maketitle

%%%%%%%%%%%%%%%%%%%%%%%%%%%%%%%%%%%%%%%%%%%%%%%%%%%%%%%%%%%%%%%%%%%%%%
\begin{abstract}
	\textit{We empirically derive a simple fourth order dynamic model of a
	bicycle. This model is then compared to various basic bicycle models derived
	from first principles. The empirical model was developed by using a
	combination of state space system identification techniques and linear
	regression on a large set of experimental data. The data was collected during
	a series of experiments on a gymnasium floor and a large treadmill. During
	the experiments, three rigidified bicycle riders manually controlled the
	bicycle using only steering control for several tasks: simple balancing, line
	tracking, and both tasks with measured lateral pulsive perturbations. The
	bicycle was instrumented to accurately measure the rider’s applied steering
	torque, all of the essential kinematics of the bicycle’s motion, and the
	lateral perturbation force. The resulting the rider control actions excited
	the system in a bandwidth adequate for identifying the linear dynamics of the
	bicycle with standard state space system identification techniques. A
	weighted linear regression was used to empirically derive relationships for
	the state and input matrices from the identified data as a function of speed.
	The resulting models were compared with several archetypal models using a
	custom interactive program allowing the comparison of statistically derived
	models to the first principles models, showing that the simplest bicycle
	models are not necessarily sufficient for capturing the fundamental dynamics
	of the vehicle. The roll acceleration was found to be reasonably predicted by
	the simpler first principles bicycle models, but the steer acceleration was
	not. A model which includes the inertial effects of the rider’s arms, which
	more closely modeled reality, improved the steer acceleration and steer
	torque predictions, but still exhibited some deficiencies. We conclude the
	paper by exploring the validity of several of the modeling assumptions and
	potential sources for the error, with a particular focus on the tire to floor
	interaction.}
\end{abstract}

%%%%%%%%%%%%%%%%%%%%%%%%%%%%%%%%%%%%%%%%%%%%%%%%%%%%%%%%%%%%%%%%%%%%%%
\begin{nomenclature}
	\entry{A}{You may include nomenclature here.}
	\entry{$\alpha$}{There are two arguments for each entry of
	the nomenclature environment, the symbol and the
	definition.}
\end{nomenclature}

%%%%%%%%%%%%%%%%%%%%%%%%%%%%%%%%%%%%%%%%%%%%%%%%%%%%%%%%%%%%%%%%%%%%%%
\section*{INTRODUCTION}

There is a rich history of bicycle and motorcycle mathematical model
development which has been able to explain many of the more dynamically
fascinating phenomena from countersteering and stability to speedman's wobble
and gyroscopic effects \cite{Limebeer2006, Meijaard2007, Meijaard2011}, but the
amount of experimental validation of these idealized models pales in
comparison, with there being more motorcycle experimental work than bicycle.

The earliest bicycle model validation began at CALSPAN in the late 60's. They
made use of a rocket to apply know step torques to an uncontrolled riderless
bicycle while measuring the roll rate and steer angle of the bicycle. The
results were compared with computer simulations in a visual manner and basic
agreement was found. TODO: read over my notes from Roland. David Eaton
identified a motorcycle with both spectral techniques and basic regression
techniques in his 1973 dissertation \cite{Eaton1973}. He did experiments with
an uncontrolled motorcycle with rididfied rider between speeds 6.7 and 20 m/s
and excited the system with handlebar taps and weight release. He compares Bode
plots and time simulations of a slight modified version of Sharp's early model
\cite{Sharp1971} with good prediction of the weave and capsize modes, but he
concluded that a different tire model was needed to predict the wobble mode. He
was also able to estimate the motorcycle model from the spectral content of the
measured steer torque and roll angle while the motorcycle was manually
controlled by the rider with a bandwidth of 1-4 rad/s. Eaton's methods have been
repeated several times over the years in various forms.

Stephen James \cite{James2002} identified an off road motorcycle which was
manually excited by the rider using black box ARX methods for a 6th and 7th
order model with the measured steer torque and the steer angle, roll rate and
yaw rate. The compared the experimental results with a 10th order motorcycle
model he developed and claimed good agreement based on visual interpretation of
the plots. He makes use of the fact that Eaton revealed that the system can be
excited through a bandwidth with manual control. Biral et al \cite{Biral2003}
constructed experimental Bode plots of a motorcycle under manual control from
steer torque and kinematic data during a series of slalom tests which forced
rider to employ sinusoidal inputs. They were able to get very good agreement
with their motorcycle model and showed how their model better predicted the
data than model's with less fidelity. Kooijman \cite{Kooijman2006,
Kooijman2009} made use of an uncontrolled riderless bicycle's stability and
excited the weave mode showing good agreement with the eigenvalues of
linearized Whipple model. His experiments have been repeated by both
\cite{Stevens2009} and \cite{Escolona2011} with some repeatability, but not
enough to negate more the need for more experimentation. Chen and Doa
\cite{Chen2010} make use of a structured black box identification of simulated
data with noise and were able to basically reproduce the eigenvalues of a
Whipple model. Doria et al \cite{Doria2012} excited the weave mode of a
motorcycle at high speed with a handlebar tap by the rider and was able to show
agreement to their motorcycle model through a phase portrait view.

Basic bicycle and motorcycle identification it typically done by exciting the
vehicle through either perturbations in roll or steer. These experiments can be
done when the bicycle is under closed loop control and/or when the bicycle is
stable, the former being require for speeds outside of the stable speed range.
The mode excitation methods are limiting because the spectrum is limited to the
frequency band around that mode of motion, making it difficult to identify
other modes and often limit the identification to less robust techniques.
Manual excitation under closed loop control allows more of the primary modes to
be identified in each test. The application of modern system identification
techniques can help make your models super duper.

\section*{MEASUREMENTS}
We constructed an instrumented bicycle with a wide variety of kinematic and
kinetic sensors. The bicycle is a heavily modified normal steel bicycle. The
rider's torso was rigidified with respect to the bicycle frame with the use of
a torso cast. The rider's legs were fixed to the frame with magnetic knee
straps and pedaling was not required by adding an electric motor for
propulsion. The rider's head and arms were able to move, the latter for
steering control. Great care was taken to measure steer torque as accurately as
possible. We made use of Futek 150 in-lb single axis torque sensor mounted such
that it was impossible to apply any forces or moments besides the steer torque.
The steer angle was measured with a potentiometer. The roll angle was measured
directly with a lightweight trailer which actuated a potentiometer. The 3
dimensional angular rates of the bicycle frame and the acceleration of a point
on the frame were measured with an inertial measurement unit. The body fixed
rate of the front frame about the steer axis was measured with a rate gyro. The
lateral perturbations delivered by a push/pull rod where measured with a 450
N load cell.

\section*{EXPERIMENTS}
We chose three riders (C, J, L) of similar age (34, 28-29, 32), mass and
bicycling ability: proficient.

We performed the experiments on the treadmill and in a large gymnasium. The
treadmill is 1 meter wide by 5 meters long and has a speed range from 0.5 m/s
to 17 m/s. The treadmill surface was made of finely grooved rubber. The
treadmill was narrow but after some practice we were able to successfully
perform the lateral perturbations experiments. This provided a very controlled
environment, in particular with respect to fixed speeds and allowed for very
long runs durations within a broad speed range.

Because all of the bicycle's data collection equipment was on board we were
able to also perform the experiments on a gymnasium floor. The floor was made
of a smooth stiff rubber and provided a rectangular wind free space of about 30
m by 55 m. We road around the perimeter to build up speed and did our maneuvers
on a straight section about 30 m long. We were not able to travel at speeds
higher than about 7 m/s due to the tires slipping in the final turn into the
test section.

Balance
   The rider is instructed to simply balance the bicycle and keep a relatively
   straight heading. They were instructed to focus one some point in the far
   distance. There was an open door infront of the treadmill which allowed the
   rider to look to a point outside. In the pavilion, the rider looked into the
   rafters of the builing or at the furthest wall. We may have given slightly
   different instructions to the riders and at least one did not understand the
   instructions exactly during some of the earlier runs, but nonetheless these
   can be analyzed with a control model that only has the roll and heading
   loops closed and maybe with only the roll loop closed. We had a line taped
   to the pavilion floor during these runs that was still in the periphery of
   the rider's vision. This may have affected their heading control.
Balance With Disturbance
   Same as 'Balance' except that a lateral force perturbation is applied just
   under the seat of the bicycle. On the treadmill, we sample for 60 to 90
   minutes with five to eleven pertubations per run. On the pavilion floor we
   were able to apply two to four perturbations per run due to the length of
   the track. In the early runs (< 204), the lateral force was applied only in
   the negative direction and the perturber decided when to apply the
   pertubations. For the later runs, we applied a random sequence of positive
   and negative perturbations. On the treadmill, the rider signaled when they
   felt stable and the perturbation was applied at a random time between 0 and
   1 second based on a simple computer program. On the pavilion floor, we
   simply applied the pertubations as soon as the rider felt stable so that we
   could get in as many as possible during each run.
Track Straight Line
   The rider was instructed to focus on a straight line that is on the ground
   and attempt to keep the front wheel on the line. The line of site from the
   rider's eyes to the the line on the ground was esentially tangent the top of
   the front wheel. In the pavilion the line could be seen up to 100 feet
   ahead, so there was a little perphiral view of the line. On the treadmill,
   there was from 0.5 to 1.5 meters of preview line available.
Track Straight Line With Disturbance
   Same as "Track Straight Line" except that a lateral perturbation force is
   applied to the seat of the bicycle. This was done in the same fashion as
   described in "Balance With Disturbance".

\section*{MODEL IDENTIFICATION}
test

\section*{FIRST PRINCIPLE MODELS}
Each rider's inertial properties were computed with a custom
software package \cite{Dembia2011} which implemented Yeadon's method
\cite{Yeadon1990a}.

\section*{MODEL COMPARISON}
test

\begin{acknowledgment}
	I'd like to thank all of the folks that have helped make these experiments
	happen. This material is based upon work supported by the National Science
	Foundation under Grant No. (grantee must enter NSF grant number).
\end{acknowledgment}

\bibliographystyle{asmems4}
\bibliography{bicycle}


\end{document}
